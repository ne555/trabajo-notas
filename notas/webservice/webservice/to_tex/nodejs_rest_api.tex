\documentclass{novela}

\title{Nodejs REST API}
\author{}
\date{}

\usepackage{listings}
\usepackage{xcolor}


\definecolor{delim}{RGB}{20,105,176}
\definecolor{numb}{RGB}{106, 109, 32}
\definecolor{string}{rgb}{0.64,0.08,0.08}

\lstdefinelanguage{json}{
    numbers=left,
    numberstyle=\small,
    frame=single,
    rulecolor=\color{black},
    showspaces=false,
    showtabs=false,
    breaklines=true,
    %postbreak=\raisebox{0ex}[0ex][0ex]{\ensuremath{\color{gray}\hookrightarrow\space}},
    breakatwhitespace=true,
    %basicstyle=\ttfamily\small,
    basicstyle=\small,
    upquote=true,
    morestring=[b]",
    stringstyle=\color{string},
    literate=
     *{0}{{{\color{numb}0}}}{1}
      {1}{{{\color{numb}1}}}{1}
      {2}{{{\color{numb}2}}}{1}
      {3}{{{\color{numb}3}}}{1}
      {4}{{{\color{numb}4}}}{1}
      {5}{{{\color{numb}5}}}{1}
      {6}{{{\color{numb}6}}}{1}
      {7}{{{\color{numb}7}}}{1}
      {8}{{{\color{numb}8}}}{1}
      {9}{{{\color{numb}9}}}{1}
      {\{}{{{\color{delim}{\{}}}}{1}
      {\}}{{{\color{delim}{\}}}}}{1}
      {[}{{{\color{delim}{[}}}}{1}
      {]}{{{\color{delim}{]}}}}{1},
}

\usepackage[framemethod=TikZ]{mdframed}
\mdfdefinestyle{codigo}{
  outerlinewidth=2pt
}
\newenvironment{codigo}
{%
  \begin{lstlisting}[language=json]
}
{%
  \end{lstlisting}
}

\begin{document}
  \frontmatter
  \maketitle
  \tableofcontents
  \mainmatter

\chapter{Arquitectura}


Este documento es una presentación de la arquitectura de la REST API.

La implementación de la REST API responderá a una arquitectura de diseño en tres
capas expresada en el siguiente diagrama:

\Image{img/000}

El diseño en tres capas es conceptual, lo que significa que en el modelo real podrá
implementarse en dos o una capas.

\section{Servidor Web}

El servidor web se implementa sobre Nodejs con un middleware de router Express.

En Express se definirán los “end points” de la API, que en principio serán siete.

La programación del server Nodejs toma datos del modelo de la Base de Datos
estructurados como “Recursos” a los cuales se les aplicarán los cuatro verbos REST.

Estos datos están almacenados como “metadata” (archivos JSON) que proveen la
información necesaria para procesar la validación de la información recibida por la API,
previo a su impacto en la Base de Datos, evitando tanto “round trips” a la base como
interpretación de errores SQL.

La API genera en forma dinámica tanto las consultas a la base como las actualizaciones
a la misma.

La API maneja las transacciones contra la base y provee control de concurrencia a nivel
fila de las tablas de la base (recursos).

La programación del Nodejs es asincrónica con manejo de “promises” y secuenciado
con “async/await”.

Las reglas de negocios se deberán programar de acuerdo a este paradigma,
devolviendo “promises”.

Estas reglas (o su invocación) deberá insertarse en el código provisto para que tenga la
capacidad de ser ejecutado en las distintas etapas del “ciclo de procesamiento”.

El servidor Web expondrá una API en formato JSON para los verbos HTTP que soporten
“body” (POST, PUT, DELETE) y un intérprete del “query string” para el verbo GET.



\section{Cliente Web}

El cliente web (fuera del alcance de este proyecto) deberá contemplar el
procesamiento asincrónico del consumo de la API interpretando los códigos de retorno
y el formato JSON del “response body”.
En el próximo envío de detallarán los “endpoints” y su funcionalidad, el formato de las
interfaces y la estructura de la Metadata.








\chapter{Metadata}


La metadata le brinda la información necesaria a la API para realizar la validación
individual y cruzada de los parámetros recibidos desde el cliente.

La API efectúa la validación individual de los parámetros respecto a tipo de dato,
longitud y cantidad de decimales, y si el parámetro es requerido o no.

Efectúa también la consistencia de “uniqness” de las columnas, validación de
existencia de valores de “foreign key” en insert y update, y chequeo de “primary key”
como “foreign key” de otras tablas en delete contemplando, si así se indica, el
“cascade delete”.

Se denomina recurso a la unidad de afectación de la API. Normalmente un recurso
tiene una relación biunívoca con una tabla de la base de datos. La excepción es el caso
de una vista (view).

La metadata se compone de un grupo de archivos JSON (dentro de la carpeta
“metadata” a partir del “root” de la aplicación). Existirá un archivo JSON por cada
recurso. El servidor lee la metadata cada vez que se instancia. Al servidor se le debe
indicar explícitamente que recursos debe leer (que archivos de metadata).

\section{Estructura de un archivo de metadata}
\begin{codigo}
{
  resource: <string> nombre del recurso,
  table: <string> nombre de la tabla en la base de datos,
  verbs: <char array> [G,P,U,D] lista de los código de verbos permitidos sobre el recuso,
  columns: <object array> array de objetos columna:
  [
    {
      name: <string> nombre de la columna en la tabla,
      rol: <string> [P, F, D, V] código de rol de la columna,
      cascade: <string> [Y,N] si el rol es “primary key” controla la acción a realizar cuando se elimina un registro: si es “N” dará un error si tiene registros de otras tablas que lo referencian, si es “Y” eliminará, en la mediad de lo posible los registros de las tablas que lo referencian,
      type: <string> [S,I,N,T,D,M,F] código de tipo de dato,
      length: <integer> longitud de la columna aplicable a tipo de dato string como longitud máxima y para columnas decimales el largo máximo total del número incluyendo punto decimal y decimales (para el resto de los tipos de dato debe ser “null”),
      decimals: <integer> cantidad máxima de decimales para el tipo de datos decimal (para otro tipo de datos debe ser “null”),
      required: <string> [Y,N] si la columna es mandatoria (“not null”),
      unique: <string> [Y,N] si el valor debe ser único en la tabla,
      table: <string> para el caso de rol “foreign key” (caso contrario “null”) indica el nombre de la tabla cuya “primary key” es “foreign key”
    }
  ]
}
\end{codigo}
\end{document}



\section{Tablas de códigos}

Verbos:

Código                                          Valor
P                                               POST
G                                               GET
U                                               PUT
Roles:

P                                             Primary Key
F                                             ForeignKey
D                                             Dato
V                                             Version control (tipo integer)


Tipos de Dato:

S                                             String
I                                             Integer
N                                             Decimal
F                                             Float (o Double)
T                                             Datetime
D                                             Date
M                                             Time
B                                             Boolean


Al instanciarse, el servidor carga la metadata y valida la existencia de los recursos
necesarios para efectuar el control de integridad referencial.


\section*{(Adenda)}

Se agrega el atributo “auto” con valores posibles “Y/N” al arreglo “columns”.

Este atributo es necesario cunado el rol = “P” (Primary key) e indica si la PK se genera
en forma automática o debe informarse.




\chapter{Instalación y set up}


\section{Instalación de Node}

Desde la página de Node (https://nodejs.org/en/download/) descargar la última
versión disponible para la plataforma adecuada (para Windows es 8.12) e instalarla.

\section{Creación del proyecto}

     -   Crear una carpeta con el nombre deseado del proyecto y hacerla corriente.

     -   Crear un proyecto con npm : “ >npm init “ completando los prompts a medida
         que se solicitan. Ninguno de los prompts es relevante para el funcionamiento
         del proyecto.

     -   Instalar las dependencias del proyecto (paquetes de node):

            o   “npm install mysql --save”         (driver de mysql)

            o   “npm install express --save”       (express router)

            o   “npm install path --save”          (path resolver)

            o   “npm install url --save”           (url resolver)

            o   “npm install body-parser --save”   (body parser)

            o   “npm install moment --save”        (date & time functions)

     -   Crear una subcarpeta “metadata”

     -   Crear una subcarpeta “src”

La estructura de carpetas deberá quedar (donde “root” es el nombre asignado a la
carpeta del proyecto):

root

..

         node_modules (carpeta)
       metadata (carpeta)

       src (carpeta)

       package.json



\section{Instalación del programa}



El programa consiste de un script principal (e-tangram.js) que deberá copiarse a la
carpeta “root”y una serie de módulos javascript que deberán copiarse a la subcarpeta
“src”.

La subcarpeta “metadata” contendrá los archivos de metadata descriptores de los
recursos.



\section{Instalación de pruebas}

Descomprimir el archivo “metadata.zip” y copiar los .json a la subcarpeta “metadata”.

Deberá crearse una base de datos (schema) con el nombre “testapi”.

Descomprimir el archivo “dbscripts.zip” y ejecutar los scripts desde el MySQL
Workbench. El archivo .zip contiene los scripts necesarios para la creación de las tablas
de ejemplo que se corresponden con los archivos .json descriptores de la metadata.

El modelo de prueba contiene cinco tablas:

   -   categorías

   -   productos

   -   clientes

   -   remitos

   -   remitos_items

las que deberán crearse en este orden para respetar la integridad referencial.

En este modelo se podrán probar los conceptos de unique, required (not null), primary
key, foreign key, cascade delete y version (concurrencia), como también los distintos
tipos de dato.

Todas las tablas tienen como PK una ID numérica (integer) autoincrementada.
Configuración:

Configuración del script “e-tangram.js”, (el server).

Editar e-tangram.js y modificar los parámetros de conexión: host, database, user y
password de acuerdo a las necesidades:

var dbConfig = {
    host:"localhost",
    database: "testapi",
    user:"sa",
    password: "mysqladmin",
    multipleStatements: true,
    supportBigNumbers: true,
    dateStrings: true,
    connectionLimit:30
}


En producción estas variables deberán guardarse en el “environment”.

El “port” por defecto donde “escucha” Node es 1337 y puede cambiarse a voluntad.

El programa e-tangram.js está ampliamente comentado.

\section{Cliente}

Para las pruebas se sugiere el uso de Postman (https://www.getpostman.com/).

\section{Nota}

4427 6845



\chapter{Descripción de la API}


\section{End Points}

La API expone 6 end points implementando los cuatro verbos standard HTTP de REST:
GET (2), POST (1), PUT (1) y DELETE (2).

Los endpoints se acceden desde el cliente con la siguientes URLs:

http://<dominio.ext>:<port>/api/<recurso>/<id>|<query>

donde:

         <dominio.ext>: nombre del dominio donde reside el server Node.

         <port>: puerto donde escucha Node (si es 80, se omite).

       <recurso>: nombre del recurso afectado por el verbo HTTP. Debe
corresponderse con el nombre del recurso en la metadata correspondiente.

         <id>: identificación de la instancia del recurso afectado.

         <query>: query string de selección de instancias del recurso.

         Nota: <id> y <query> son mutuamente excluyentes.

\section{HTTP GET}

Un endpoint para acceso con “id” y otro para acceso con “query”.

Ejemplos:

http://miserver:1337/api/cliente/3

recupera la instancia del recurso con clienteId = 3

http://miserver:1337/api/item/? remitoItemCantidad ge 12 &remitoItemCantidad le
24

recupera las instancias con cantidad entre 12 y 24

\section{HTTP POST}

Un endpoint donde se “postea” el contenido del “body”:

El “body” será un objeto JSON (stringified) que contendrá las columnas y los valores a
insertar.

Ejemplo:

http://miserver:1337/api/producto

body:

{

“productoCategoriaId”: 2,

“productoNombre”: “Envase plástico”,

“productoDescripcion”: ”Descripción de envase plástico”

}

Crea una nueva instancia del recurso producto con los valores indicados de las
columnas.



\section{HTTP PUT}

Un endpoint para selección por “id” de la instancia del recurso a modificar.

Ejemplo:

http://miserver:1337/api/cliente/3

body:

{

“clienteNombre”: “Mi Nuevo nombre”,

“clienteVersion”: 325

}

Actualiza el nombre del cliente en la instancia clienteId = 3 si la versión de la instancia
del recurso es 325. Caso contrario no la actualiza.

\section{HTTP DELETE:}

Un endpoint para selección por “id” (una instancia) y otro para selección con “query”
(múltiples instancias).

Ejemplos:

http://miserver:1337/api/cliente/3

elimina la instancia del recurso cliente con clienteId = 3

http://miserver:1337/api/item/? remitoItemCantidad ge 12 &remitoItemCantidad le
24

elimina las instancias del recurso remito con cantidad entre 12 y 24



\section{Retorno}

El código de retorno HTTP de la API es siempre 200 (OK).

El body del HTTP Response es un objeto JSON (stringified) con dos elementos:

returnset, que contiene información de resultado de la operación, y dataset, que
contiene los datos recuperados.

returnset es un arreglo con una sola ocurrencia de un objeto y tiene contenido en las
respuestas de las cuatro operaciones.

dataset es un arreglo con tantas ocurrencias como “registros” se hayan recuperado y
tiene contenido solo en la respuesta de la operación GET.

{

“returnset”: [{

       “RCode”:<integer> (código de retorno de la operación),

       “RTxt”:<string> (texto de retorno de la operación),

       “RId”: <integer> (id de la nueva instancia, solo POST),

       “RSQLErrNo”:<integer> (retorno de MySQL),

       “RSQLErrtxt”:<string> (texto de retorno de MySQL)

       }],

“dataset”: [{}]

}

\section{Nota}

La descripción y sintaxis del query y la tabla de códigos de retorno en la próxima
entrega.

4427 6845




\chapter{Descripción de la API (query string)}


El “query string” se utiliza en los verbos GET y DELETE como alternativa de selección de
las filas a ser afectadas por cada sentencia.

En ambos verbos también existe la selección por Id del recurso, como se indicó en la
parte 4 de este documento.

El “query string” consiste en una o más ocurrencias de pares nombre/valor separados
por el signo “&”. El “nombre” se separa del “valor” con el signo “=”.

Estos pares nombre/valor son analizados por la API para fijar los siguientes
parámetros:

   -   Cláusula where.

   -   Cláusula order by.

   -   Meta instrucciones include/exclude.

   -   Limit.

   -   Offset.

\section{Cláusula where}

Consiste en una cantidad ilimitada de pares nombre/valor:

   -   El nombre indica el nombre de la columna del recurso referenciado, de acuerdo
       a su “name” en la metadata.

   -   El valor consiste de dos elementos:

           o     Operador de comparación/inclusión.

           o     Valor, necesario para la comparación/inclusión. El valor null no es
                 permitido. Para checkear por null utilizar los valores especiales “isnull” e
                 “isnotnull”

\subsection{Operadores}
Los operadores son los siguientes:

   -   “eq”, igualdad.

   -   “not”, desigualdad.

   -   “lt”, menor que.

   -   “le”, menor o igual que.

   -   “gt”, mayor que.

   -   “ge”, mayor o igual que.

   -   “lk”, like, ídem SQL.

   -   “in”, in a set, ídem SQL.

\subsection{Valores:}

El valor de comparación debe indicarse entre corchetes “[“ “]” y deberá ser del mismo
tipo que el definido en la metadata para la columna en cuestión.

Ejemplos:

http://miserver:1337/api/item/? remitoItemCantidad = ge [12] &remitoItemCantidad
= le [24]

http://miserver:1337/api/cliente/? clienteNombre = lk [ba%]

http://miserver:1337/api/cliente/? clienteId = in [1,3,4]

http://miserver:1337/api/producto/? productoDescripcion = eq [isnotnull]

selecciona aquellos productos cuya descripción no es “null”.




\section{Cáusula orderby (sólo GET)}

Puede especificarse o no. Si se especifica debe ser sólo una vez.

   -   El nombre es la palabra clave “_orderby”

   -   El valor es una serie de elementos columna/orden separados por coma.

            o   columna es el nombre de la columna en la metadata.

            o   orden, ordenamiento ascendente “A” o descendente “D”, default “A”
            La columna y el orden deben estar separados por “espacio”.

Ejemplos:

http://miserver:1337/api/cliente/? clienteNombre          = lk [ba%] & _orderby =
clienteNombre A, clienteId D

ascendente por nombre del cliente y ,por igualdad, descendente por Id del cliente

http://miserver:1337/api/remito/? _orderby = remitoFecha

ascendente por fecha de remito (default Asc).



\section{Meta include/exclude (sólo GET)}

Pueden especificarse o no. Son excluyentes.

Son utilizados para limitar las columnas seleccionadas.

Si no se informan se recuperan todas las columnas del recurso (tabla).

Utilizar “_include” para indicar que columnas se seleccionan.

Utilizar “_exclude” para indicar que columnas no se seleccionan, a partir de todas las
columnas del recurso (tabla).

   -   El nombre es la palabra clave(“_include” / ”_exclude”).

   -   El valor es una lista de los nombres de las columnas (de acuerdo a la metadata)
       separados por coma, que se incluyen / excluyen en la selección.

Ejemplos:

http://miserver:1337/api/cliente/? _exclude = clienteVersion, clienteId

excluye clienteVersion y cliente Id de la selección.

http://miserver:1337/api/cliente/? _include = clienteNombre

selecciona solamente la columna clienteNombre



\section{Meta limit (GET)}

Envía el valor de la cláusula limit a la sentencia SQL.

   -   El nombre es la palabra clave “_limit”.
   -   El valor es un número entero que indica la cantidad de filas a recuperar.

Ejemplo:

http://miserver:1337/api/remito/? remitoFecha = ge [2018-09-01] & _limit = 20

selecciona las 20 primeras líneas de remitos que cumplen con la condición de fecha
mayor que el 31/08/2018.

\section{Meta offset (GET)}

Envía el valor de la cláusula offset a la sentencia SQL.

   -   El nombre es la palabra clave “_offset”.

   -   El valor es un número entero que indica a partir de que fila se va a recuperar.

Ejemplo:

http://miserver:1337/api/remito/? remitoFecha = ge [2018-09-01] & _limit = 20 &
_offset = 20

selecciona 20 primeras líneas de remitos que cumplen con la condición de fecha mayor
que el 31/08/2018, a partir del registro 21.

Se utilizan para paginados.





\chapter{Descripción de la API (códigos de retorno)}


Como se indicó en la parte 4, la API devuelve un response body con la siguiente
estructura:

{

“returnset”: [{

       “RCode”:<integer> (código de retorno de la operación),

       “RTxt”:<string> (texto de retorno de la operación),

       “RId”: <integer> (id de la nueva instancia, solo POST),

       “RSQLErrNo”:<integer> (retorno de MySQL),

       “RSQLErrtxt”:<string> (texto de retorno de MySQL)

       }],

“dataset”: [{}]

}

El código de retorno HTTP de la API es siempre 200 (OK).

El body del HTTP Response es un objeto JSON (stringified) con dos elementos:

returnset, que contiene información de resultado de la operación, y dataset, que
contiene los datos recuperados.

returnset es un arreglo con una sola ocurrencia de un objeto y tiene contenido en las
respuestas de las cuatro operaciones.

dataset es un arreglo con tantas ocurrencias como “registros” se hayan recuperado y
tiene contenido solo en la respuesta de la operación GET.



RCode: return code, es el resultado de la operación, puede tener los siguientes valores:

       -     Éxito, se corresponde con el valor 1 (uno). (RTxt: “OK”)
      -   Error MySQL, se corresponde con el valor 0 (cero). (Rtxt: “ErrorMySQL”). En
          este caso los campos RSQLErrNo y RSQLErrTExt contienen el número de
          error y el texto del mismo tal cual fueron devueltos por MySQL.

      -    RCode < 0. En este caso es un error específico de la API. RTxt continene la
          explicación del mismo. Los valores posibles se agrupan en 3 rangos
          distintos:

              o     -1xxx, corresponden a errores de sintaxis y validación de los
                    parámetros enviados contra la metadata.

              o     -2xxx, corresponden a errores de la lógica de actualización de la
                    base.

              o     -5xxx, corresponden a errores internos inesperados.

              La lista de los errores se detalla a continuación.

      -   RId: es la Id asignada por la base cuando ésta es “auto: Y” y se produce una
          inserción.

Tabla de Códigos:

Return       Text
     -1000   JSON Malformado
     -1001   Recurso inválido
     -1002   Verbo no soportado en el recurso
     -1003   Body vacío
     -1004   Nombre de miembro inválido en body
     -1005   PK no informada, no auto
     -1006   Version no informada
     -1007   Columna requerida no informada
     -1008   Tabla referenciada (FK), no encontrada en metadata
     -1009   Tabla sin PK
     -1010   Pk referenciada es de distinto tipo que la Fk referenciante
     -1011   Valor inválido para columna tipo boolean
     -1012   Valor inválido para columna tipo date/time/datetime
     -1013   Valor no numérico para columna numérica
     -1014   Valor no entero para columna entera
     -1015   Cantidad de decimales excedida para columna decimal
     -1016   Longitud de columna string excedida
     -1017   _include y _exclude son excluyentes
     -1018   _include no permitido en DELETE
     -1019   _include vacío
     -1020   Nombre de columna inválido
     -1021   _exclude vacío
     -1022   _orderby no permitido en DELETE
-1023   _orderby vacío
-1024   _orderby, error de sintaxis
-1025   _orderby, tipo de orden inválido
-1026   _orderby, columna no seleccionada
-1027   Corchetes desbalanceados
-1028   Los corchetes no se pueden anidar
-1029   Valor de query no informado
-1030   Operador de query inválido
        Null no puede ser usado como parámetro, utilizar 'isnull' o
-1031   'isnotnull'
-1032   Tipo de dato inválido
-1033   Tipo de dato inválido en lista
-1034   _offset no permitido en DELETE
-1035   _offset debe ser numérico
-1036   _offset inválido
-1037   _limit no permitido en DELETE
-1038   _limit debe ser numérico
-1039   _limit inválido
-2001   Valor duplicado para columna unique
-2002   Valor de FK no encontrado
-2003   No encontrado
-2004   Versiones distintas
-2005   PK con dependencias como FK
-5001   Error interno buscando columna en metadata
-5002   Tabla referenciada no encontrada en metadata
-5003   PK no encontrada en metadata
-5004   Version no encontrada en metadata
-5005   Columna no encontrada en metadata




\chapter{Descripción de la API (Concurrencia)}


\section{Concurrencia:}

El control de concurrencia en la operación update se realiza con el mantenimiento de
una columna de “versión” de la fila. Al crearse una nueva fila el valor de su versión es
cero.

La estrategia consiste en leer el registro correspondiente con su versión (ordinal) y al
ejecutar el update se verifica que la versión sea la misma que la leída.

Si no es la misma la API devuelve el código -2004 (versiones distintas).

Si es la misma se produce el update y la versión es incrementada en uno.

La columna que mantiene la versión en la tabla debe decribirse con rol : “V” en la
metadata.




\chapter{Uso de vistas (views)}


La API interpreta (como SQL) a las vistas como tablas de “solo lectura”.

Para su uso:

   1. Crear la vista con su nombre.

   2. Crear un archivo de metadata con la misma estructura que un archivo
       metadata de tabla asignándole al atributo “table” el nombre de la vista y
       describiendo las columnas que serán seleccionadas por defecto (esto luego
       podrá ser afectado por “_exclude /_include”). En el arreglo “verbs” colocar solo
       GET (G). Ej: “verbs”: [“G”]

   3. Agregar el archivo de metadata con su sentencia “required” correspondiente
       en “e-tangram.js” (línea 36 aprox.)

   4. Reiniciar el server.

   5. Utilizar el GET con selección, ordenamiento, etc.

   6. Se adjunta un ejemplo sobre la base de prueba.




\chapter{JWT (JSON Web Token)}


\section{Atención}

La instalación de esta versión requerirá cambios en los requests a la API.

JSON Web Token provee un método sencillo de intercambio de requests con la API una
vez efectuado el Log In.

La implementación es un “middleware” que se procesa en cada request.

\section{Mecanismo}

   1. El cliente envía un log in (usuario/password) en el “body” de un POST.

   2. El server recibe, valida y genera un JWT utilizando un “payload” y la clave
       “secreta”. Este token será válido por un determinado tiempo, expresado en
       segundos.

   3. El server envía el token al cliente.

   4. Para todo request el cliente debe enviar el token recibido en el header de dicho
       request con key “x-access-token” y value <token>.

   5. El server recibe el request, verifica la presencia y validez del JWT y decodifica el
       “payload”.

   6. En caso de ser inválido, el server retorna un código de error (-6002 / -6003).

   7. Si es exitoso el middleware inserta el “payload” decodificado en el request y
       continúa el procesamiento del mismo(next()).

\section{Implementación}

Deberá instalarse los packages:

npm install jsonwebtoken --save

npm install bcrypt --save

Se ha agregado el endpoint http://<miserver>/login para el verbo POST.
El mismo espera un JSON



{

       “usuario”: <string>

       “pass”: <string>

}

El endpoint valida el contenido contra un arreglos de objetos (usuarios, línea 33) y si el
par usuario/password es válido genera un JWT con payload = usuarioId y clave secreta
JWTSecret (línea 30), devolviendo al cliente el jwt en el dataset.

Los request sucesivos a cualquier endpoint serán procesados por el
“middleware” (rest_token.js). De ser correcto, insertará el valor del
payload (usuarioId) en el request y continuará la ejecución de acuerdo a
lo programado en cada endpoint.

En este ejemplo se puede ver como, a partir de la validación y del encriptado del
payload, puede obtenerse el valor de usuarioId sin necesidad de acceder a la DB.



\section{Mensajes}

Se agregan los siguientes códigos de retorno:

       Código         Texto

            -6001 Usuario/Password inválido
            -6002 Debe proveerse un token
            -6003 Token inválido




\chapter{Stored Procedures Bridge}


\section{Objetivo}

Se agrega una nueva funcionalidad a la API que consiste en la capacidad solicitar
ejecuciones de Stored Procedures de la base de datos.



\section{Implementación}

Se ha creado un nuevo endpoint para procesar estos “request”:

http://<miserver>/api/proc/<nombre>

El verbo a ejecutar es POST.

Los parámetros de ejecución se envían en el body.

<nombre>: nombre del “servicio” definido en la metadata.



\section{Metadata}

Como los “recursos” aplicables a la REST API, los stored procedures también requieren
metadata para efectuar la validación de los parámetros recibidos y el ordenamiento de
los mismos en la invocación al SP (recordemos que los parámetros de un llamado a un
SP son posicionales).

La diferencia importante con la metadata de “recursos” es que los parámetros no
necesitan corresponderse con nombres de columnas de tablas de la base de datos.

Cada archivo de metadata para stored procedures puede contener “n” definiciones de
parámetros, es decir metadata para más de un SP. Cada archivo comprende
“lógicamente” a los procedimientos que afecten a un recurso en un sentido amplio.

La metadata para SPs es un archivo JSON con el siguiente formato:
{
    "resource”:<string> (nombre del recurso)
    "sps":<array> (de definiciones de SP)
       [
         {
              "servicio": <string> (nombre “externo” del SP),
              "type": <string> “C”,”R”,”U”, “D” (tipo de operación del SP)
              "sp": <string> (nombre “interno”, en la base, del SP),
              "params": <array> (de parámetros ordenados, para ejecución)
              [
                  {
                    "name":<string> (nombre del parámetro),
                    "required": <string> “Y”/”N” (si es requerido o no)
                  }, …
             ]
         }, …
    ],
    "columns":<arreglo> (de descriptores de columnas, parámetros)
       [
         {
              "name": <string> (nombre del parámetro)
              "type": <string> (tipo de dato, idem tablas),
              "length": <int> (longitud),
              "decimals": <int> (decimales
         },…
       ]
}

El arreglo “columns” permite las definiciones de tipo y longitud para los parámetros,
mientras que el arreglo “params” indica, para cada stored procedure, que columna
debe incluirse, en qué orden y la obligatoriedad de informarlo en el request.

Se adjunta un ejemplo de metadata para cuatro SPs sobre la tabla “producto”
(producto_sp.json).

Los archivos de metadata de definición de SPs residirán en la misma carpeta
(.\metadata) que los archivos de metadata de tablas.



\section{Carga de Metadata}

Esta versión cambia el procedimiento de carga de la metadata, eliminando la
necesidad de tocar el código del server cada vez que se agrega o quita un archivo de
metadata.

Para esto se crea un catálogo de metadata en la misma carpeta (.\metadata), con el
nombre “meta_catalogo.json”.
Este archivo de catálogo consiste de un arreglo bajo el nombre “catalog” donde cada
elemento es un objeto con dos miembros:

   -   name: nombre del archivo de metadata a cargar (sin extensión)

   -   type: tipo de metadata (“T”:tabla, “V”:vista(view) y “S”:stored procedure)

Una vez creada la metadata para una tabla, vista o grupo de SPs, deberácrearse la
entrada en el catálogo para que sea cargado en el server, cada vez que éste se reinicie.

Se envía el ejemplo actualizado para el conjunto de ejemplos.



\section{De los Stored Procedures}

Los SPs deberán construirse con una estructura de códigos de retorno consistentes con
el esquema general de retorno de la API, tanto para que la misma sepa en qué orden
retornar estos códigos como para que la aplicación cliente pueda dialogar de una
forma independiente de los procesos subyacentes.

Se adjuntan cuatro Stored Procedures (Insert, Delete, Update y Retrieve) para la tabla
“productos” como ejemplo de estructura y retornos.

Estos SPs son coincidentes con el archivo de definición de los mismos
(producto_sp.json)

\end{document}
